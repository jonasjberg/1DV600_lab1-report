% ______________________________________________________________________________
%
%   1DV600 - Software Technology
%   Assignment 1 -- "Personal planning, Vision and Project Plan"
%
%  Author:  Jonas Sjöberg
%           Linnaeus University
%           js224eh@student.lnu.se
%           https://github.com/jonasjberg
%
%    Date:  2017-02-01 -- 2017-02-05
%
% License:  Creative Commons Attribution 4.0 International (CC BY 4.0)
%           <http://creativecommons.org/licenses/by/4.0/legalcode>
%           See LICENSE.md for additional licensing information.
% ______________________________________________________________________________


\section{Assignment overview}
% TODO: ..

This is the report for the first assignment in the course \texttt{1DV600} --
``Software Technology'' given at Linneaus University during the spring of 2017.


% ______________________________________________________________________________
\subsection{Background}
% TODO: ..



% ______________________________________________________________________________
\subsection{Purpose}
% TODO: ..


% ______________________________________________________________________________
% \subsection{Method}
% 
% Nedan följer en redogörelse för den arbetsmetod som användes för utveckling av
% webbplatsen och framställning av all relaterad dokumentation:
% 
% \begin{itemize}
%   \item Operativsystemet som används är Linux, kernel version
%       \texttt{4.4.0-42-generic}.
% 
%   \item Rapporten skrivs i \LaTeX\  som kompileras till pdf med
%       \texttt{latexmk}.
% 
%   \item Både rapporten och eventuell kod skrivs med texteditorn \texttt{Vim}
%       och utvecklingsmiljön \texttt{Intellij Webstorm 2016.2}.
% 
%   \item För versionshantering av både rapporten och programkod används
%       \texttt{Git}.
% 
%   \item Vid förhandsgranskning används primärt \texttt{Google Chrome} w
%       version \texttt{53.0.2785.92 (64-bit)} och \texttt{Mozilla Firefox 49.0}.
%     % \begin{itemize}
%     %   \item Källkod till programmet och rapporten finns att hämta på:
% 
%     %         \url{https://github.com/jonasjberg/1dv600-lab}
% 
%     %   \item Hämta hem repon genom att exekvera följande från kommandoraden:
%     %         
%     %         \texttt{git clone git@github.com:jonasjberg/1dv600-lab.git}
% 
%     % \end{itemize}
% \end{itemize}


