% ______________________________________________________________________________
%
%   1DV600 - Software Technology
%   Assignment 1 -- "Personal planning, Vision and Project Plan"
%
%  Author:  Jonas Sjöberg
%           Linnaeus University
%           js224eh@student.lnu.se
%           https://github.com/jonasjberg
%
%    Date:  2017-02-01 -- 2017-02-05
%
% License:  Creative Commons Attribution 4.0 International (CC BY 4.0)
%           <http://creativecommons.org/licenses/by/4.0/legalcode>
%           See LICENSE.md for additional licensing information.
% ______________________________________________________________________________


% ______________________________________________________________________________
\section{Task 1 -- Personal Planning}

\subsection{Instructions}\label{task-1-instructions}
Instructions quoted as-is from the course Wiki \cite{1dv600:lab1:instructions}.

\begin{quote}
  When the client requests a list of books to present for the user it does the
  call \texttt{http://localhost:9090/api/books/} to the server and it expects
  the answer as a JSON object (an associative array). We are going to split the
  functionality into three tasks, but it is your task to plan these tasks. Take
  into account the time for learning and understanding of the problem when you
  plan the time. Make your planning with 15 minutes as the minimum unit. Repeat
  the following pattern for all subtasks (A, B, C):
  
  \begin{itemize}
    \tightlist
    \item Plan
    \item Implement
    \item Reflect
  \end{itemize}
  
  Each subtask should be documented with at least 100 words.
\end{quote}


% ______________________________________________________________________________
\section{Subtask A -- Books}\label{task-1a}
\subsection{Instructions}\label{task-1a-instructions}

\begin{quote}
  The main objective of the subtask is to create a list of books and a function
  or method to get them. There are slight differences depending on which
  implementation you are using, either Java or Node.js but those differences
  will be clearly noted. Common for both is that they should handle books,
  and for each book we need the information id, title, author, genre, publish
  date, price and description.

  \subparagraph{Java}
  Create a class in the package ``models'' that represents a book. After
  that, create a short list of fictive (or real) objects in the function
  getBooks that is available in GetBooksResource. When calling the URL
  http://localhost:9090/api/books the list of books should be outputted
  with System.out.println. The subtask is done when you see the objects in
  the terminal (where vagrant is run).
\end{quote}


\subsection{Plan}\label{task-1a-plan}
The \texttt{book} class is a critical datatype that will be used throughout the
software. As such, it would probably be wise to design the class in a way as 
to make it flexible and open for future modifications, as stated by the well known
``open/closed principle'' \cite{SOLID:OCP}, briefly stated \cite{SOLID:OCP:Meyer}:

\begin{quote}
  software entities (classes, modules, functions, etc.) should be open for
  extension, but closed for modification
\end{quote}


The class \texttt{book} should meet the following requirements:

\begin{itemize}
\tightlist
  \item Store information needed to describe books handled by the system.
        We are to include the following set of attributes:

  \begin{itemize}
    \tightlist
    \item Id
    \item Title
    \item Author
    \item genre
    \item Date of publication
    \item Price
    \item Description
  \end{itemize}

  \item Provide suitable interfaces for accessing and possibly modifying the 
        data encapsulated in the class.
\end{itemize}

The initial plan is to implement a basic class with traditional mutators for
accessing private fields. This first implementation will use primitive data
types for its fields, which often means future expansion and modifications will
require refactoring. 

This design opposes the previously mentioned ``open/closed principle'', but
future refactoring seems a reasonable tradeoff in order to get a prototype
working as soon as possible.

This implementation should use the least amount of code possible to meet the
requirements.

% TODO: Finish this section
% We will have to refactor the code at some point in the future.

% Wrapper data types would be better


\subsection{Implement}\label{implement}

\subsection{Reflect}\label{reflect}


% ______________________________________________________________________________
\section{Subtask B -- JSON}\label{subtask-b-json}

\subsection{Instructions}\label{instructions-2}

\begin{quote}
Convert the objects created in subtask a into an JSON object and show it
in the terminal using either System.out.println(Java) or console.log
(Node.js).

\mbox{}%
\subparagraph{Improvement Strategies}\label{improvement-strategies}

Choose two improvement strategies based on your reflections on subtask a
and b. Describe what you have decided to improve and why. Implement your
improvements in the next subtask.
\end{quote}


% ______________________________________________________________________________
\section{Subtask C -- Web}\label{subtask-c-web}

In this subtask you are to answer the request in the web browser instead
of printing it to the terminal. The subtask is done when you see the
JSON object on screen. For inspiration, have a look at PingResource that
you find in the same folder as the GetBooksResource. If you follow the
\href{https://htmlpreview.github.io/?https://github.com/tobias-dv-lnu/1dv600-lab/blob/master/api-specification/api-specification.html}{API
for the model} (as seen in
\href{https://htmlpreview.github.io/?https://github.com/tobias-dv-lnu/1dv600-lab/blob/master/api-specification/api-specification.html\#books-get}{GET
api/books})​, you will be able to show the books in the list.

