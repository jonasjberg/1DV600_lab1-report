% ______________________________________________________________________________
%
%   1DV600 - Software Technology
%   Assignment 1 -- "Personal planning, Vision and Project Plan"
%
%  Author:  Jonas Sjöberg
%           Linnaeus University
%           js224eh@student.lnu.se
%           https://github.com/jonasjberg
%
%    Date:  2017-02-01 -- 2017-02-05
%
% License:  Creative Commons Attribution 4.0 International (CC BY 4.0)
%           <http://creativecommons.org/licenses/by/4.0/legalcode>
%           See LICENSE.md for additional licensing information.
% ______________________________________________________________________________


% ______________________________________________________________________________
\section{Task 1 -- Personal Planning}
\paragraph{Instructions}\label{task-1-instructions}
from the course Wiki\cite{1dv600:lab1:instructions}:

\begin{quote}
  When the client requests a list of books to present for the user it does the
  call \texttt{http://localhost:9090/api/books/} to the server and it expects
  the answer as a JSON object (an associative array). We are going to split the
  functionality into three tasks, but it is your task to plan these tasks. Take
  into account the time for learning and understanding of the problem when you
  plan the time. Make your planning with 15 minutes as the minimum unit. Repeat
  the following pattern for all subtasks (A, B, C):

  \begin{itemize}
    \item Plan
    \item Implement
    \item Reflect
  \end{itemize}

  Each subtask should be documented with at least 100 words.
\end{quote}


% ______________________________________________________________________________
\subsection{Subtask A -- Books}\label{task-1a}
\paragraph{Instructions}\label{task-1a-instructions}
from the course Wiki\cite{1dv600:lab1:instructions}:

\begin{quote}
  The main objective of the subtask is to create a list of books and a function
  or method to get them. There are slight differences depending on which
  implementation you are using, either Java or Node.js but those differences
  will be clearly noted. Common for both is that they should handle books,
  and for each book we need the information id, title, author, genre, publish
  date, price and description.

  \subparagraph{Java}
  Create a class in the package ``models'' that represents a book. After
  that, create a short list of fictive (or real) objects in the function
  getBooks that is available in GetBooksResource. When calling the URL
  http://localhost:9090/api/books the list of books should be outputted
  with System.out.println. The subtask is done when you see the objects in
  the terminal (where vagrant is run).
\end{quote}


\subsubsection{Plan}\label{task-1a-plan}
The \texttt{book} class is a critical datatype that will be used throughout the
software. As such, it would probably be wise to design the class in a way as to
make it flexible and open for future modifications, as stated by the well known
``open/closed principle''\cite{Martin:1996:OPS} -- in brief\cite{SOLID:OCP:Meyer}:

\begin{quote}
  Software entities (classes, modules, functions, etc.) should be open for
  extension, but closed for modification.
\end{quote}

In practical terms, this would mean avoiding primitive data types and instead
creating ones own custom data types for the metadata fields; author, title,
etc.


The class \texttt{book} should meet the following requirements:

\begin{itemize}
  \item Store information needed to describe books handled by the system.
        We are to include the following set of attributes:

  \begin{itemize}
    \item Id
    \item Title
    \item Author
    \item genre
    \item Date of publication
    \item Price
    \item Description
  \end{itemize}

  \item Provide suitable interfaces for accessing and possibly modifying the
        data encapsulated in the class.
\end{itemize}

The initial plan is to implement a basic class with suitable private fields for
storing data about a certain book, as well as traditional mutators for
accessing private fields.

This first implementation will use primitive data types for its fields, which
often means future expansion and modifications will require refactoring.
This design choice goes against the previously mentioned ``open/closed
principle'', but future refactoring seems a reasonable trade-off in order to get
a prototype working as soon as possible.


In other words: this implementation should use the least amount of code
possible to meet the requirements.


\subsubsection{Implement}\label{task-1a-implement}
The class was implemented in less than 10 minutes. What followed was
investigating bugs in the underlying framework
``dropwizard''\cite{framework:dropwizard}.  
The problems seems to be caused by printing a combination of newlines and other
characters to ``stdout'', which is hooked by either ``dropwizard'' or the build
system ``gradle''\cite{tool:gradle}, in order to add additional information and
timestamps to the message.  An error occurs somewhere along this chain.


\subsubsection{Reflect}\label{task-1a-reflect}
This simple implementation matches the criteria which seems vague at this point,
the \texttt{book} class presents its state in \texttt{JSON}-format, and so deciding on
implementation details becomes very difficult. These details are for example
the internal data types in the ``book'' class and the overall program
structure, including communication between different parts of the program which
again uses some kind of data types.
These details grow to reach ever larger parts of the source code as the software
grows in size and complexity.
I know for a fact that this simple implementation will need serious rework
in order to be extended. But making the design more involved and modular
at this stage, with no information to go on, could possibly be a waste of time.

This kind of trade-off seems to reoccur and turn up in various forms.
The trade-off is between the following two extremes:

\begin{description}
  \item [A] Try to predict future usage and development, design for maximum
        flexibility and modularity. Means more initial work is needed to reach
        a minimum working state, but can pay off in the long run, the design
        makes modification and extension easier.  The risk is that the extra
        functionality and complexity might go complete unused, and the whole
        enterprise is thus a complete waste of time.
  \item [B] Assume very little about future developments, use the simplest
        possible design that meets the requirements. Accept that the simple
        solution is more rigid and sensitive to changes --- future changes in
        requirements or functionality will require major refactoring.
\end{description}


% ______________________________________________________________________________
\subsection{Subtask B -- \texttt{JSON}}\label{task-1b}
\paragraph{Instructions}\label{task-1b-instructions}
from the course Wiki\cite{1dv600:lab1:instructions}:

\begin{quote}
  Convert the objects created in subtask a into an \texttt{JSON} object and show it in
  the terminal using either System.out.println(Java) or console.log (Node.js).

  \subparagraph{Improvement Strategies}

  Choose two improvement strategies based on your reflections on subtask a and
  b. Describe what you have decided to improve and why. Implement your
  improvements in the next subtask.
\end{quote}


\subsubsection{Plan}\label{task-1b-plan}
As with the first subtask, the plan is to use the simplest possible solution
that meets the requirements.
The first implementation will use simple, error-prone and crude raw string
manipulations to construct \texttt{JSON}-data.
If this method performs as expected when integrated into the rest of the system,
methods for converting from Java objects to \texttt{JSON}-data could be refactored into
a separate utility package.


\subsubsection{Implement}\label{task-1b-implement}
The code was written in less than 5 minutes. The basic method for
constructing the strings was re-used from the previous subtask.

The method that returns the \texttt{JSON} data, ``\texttt{toJSON()}'' is shown
in Listing~\ref{listing:tojson1}.

\javasource{include/tojson1.java}
           {Initial implementation of the \texttt{toJSON\(\)} method in the
            \texttt{Book} class.}
           {listing:tojson1}


\subsubsection{Reflect}\label{task-1b-reflect}
This should really be solved using the standard libraries native functions for
handling \texttt{JSON}.
At the very least, these methods should be extracted and refactored into
utility functions stored in a separate package with other similar ``utility''
code.  This enables re-use across the entire project. Many companies re-use
utility code across many if not all projects.


\paragraph{Improvement Strategies}
Based on the reflections on Subtask A \ref{task-1a-reflect} and Subtask B
\ref{task-1b-reflect}, the two following improvements was chosen:

\begin{enumerate}
  \item Improve \texttt{JSON} handling by leveraging the Java libraries.
  \item Create classes to wrap primitive data types -- add containers for the
        metadata fields such as author, title, etc.
\end{enumerate}



% ______________________________________________________________________________
\subsection{Subtask C -- Web}\label{task-1c}
\paragraph{Instructions}\label{task-1c-instructions}
from the course Wiki\cite{1dv600:lab1:instructions}:

\begin{quote}
  In this subtask you are to answer the request in the web browser instead of
  printing it to the terminal. The subtask is done when you see the \texttt{JSON} object
  on screen. For inspiration, have a look at PingResource that you find in the
  same folder as the GetBooksResource.

  If you follow the
  \href{https://htmlpreview.github.io/?https://github.com/tobias-dv-lnu/1dv600-lab/blob/master/api-specification/api-specification.html}{API for the model}
  (as seen in
  \href{https://htmlpreview.github.io/?https://github.com/tobias-dv-lnu/1dv600-lab/blob/master/api-specification/api-specification.html\#books-get}{GET api/books})
  , you will be able to show the books in the list.
\end{quote}


\subsubsection{Plan}\label{task-1c-plan}
As per the instructions, the plan is to look for the existing code to hints
on how to meet the requested functionality.

The chosen improvement strategies will be solved by using the suggested JSON
library ``jackson''\cite{jackson-json}.

\subsubsection{Implement}\label{task-1c-implement}
The implementation was done as part of experimenting with the previous two
subtasks and the strange behaviour that the framework ``dropwizard'' exhibited
when new-lines was passed to the terminal.

The library ``jackson'' now handles \texttt{JSON} conversion,  this revised
method ``\texttt{toJSON()}'' is shown in Listing~\ref{listing:tojson2}.

\javasource{include/tojson2.java}
           {Revised implementation of the \texttt{toJSON\(\)} method in the
            \texttt{Book} class.}
           {listing:tojson2}

The book metadata field ``genre'' in the \texttt{Book} class was improved by
using an \texttt{enum} instead of a string. This adds type checking as the
compiler will catch any errors at compile-time -- the genre must be set to one
of the predetermined set of possible values. The \texttt{enum} type also opens
up for future addition of other functionality.


\subsubsection{Reflect}\label{task-1c-reflect}
The modifications done to the class was minor but did improve the overall
quality and robustness of the code.

All the metadata fields should probably completely wrap their contained data.
Possibly, an abstract class called ``MetadataField'' could be added. This class
would have functionality for setting and getting the metadata value in a safe
and controlled manner.

Specific fields, for example ``author'' could also be stored in a class
``AuthorMeta'' that inherits from the ``MetadataField'' class. These subclasses
would then add their own behaviour, specific to their contained metadata.

This would allow future expansion and handling of special cases like multiple
authors, etc.
