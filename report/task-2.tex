% ______________________________________________________________________________
%
%   1DV600 - Software Technology
%   Assignment 1 -- "Personal planning, Vision and Project Plan"
%
%  Author:  Jonas Sjöberg
%           Linnaeus University
%           js224eh@student.lnu.se
%           https://github.com/jonasjberg
%
%    Date:  2017-02-01 -- 2017-02-05
%
% License:  Creative Commons Attribution 4.0 International (CC BY 4.0)
%           <http://creativecommons.org/licenses/by/4.0/legalcode>
%           See LICENSE.md for additional licensing information.
% ______________________________________________________________________________


% ______________________________________________________________________________
\section{Task 2 -- Vision}\label{task-2}

\paragraph{Instructions}\label{task-2-instructions}
from the course Wiki\cite{1dv600:lab1:instructions}:

\begin{quote}
  Create a vision document for the system. This should be a document covering
  about half an A4 page describing the system. The purpose of the document is
  to make sure that everyone involved in the project has the same vision of
  what is to be created. Use the ``Assignment Overview'' and previous subtasks
  as your source for what to write. In addition, write down your reflections on
  creating a vision document. This reflection should be about 100 words.
\end{quote}


% ______________________________________________________________________________
\subsection{Vision Document}\label{task-2-vision}
The project described is a web-application for managing a collection of books.
This application is the foundation for a library system for books where you can
add, modify and delete books.


\subsubsection{Intended use}
The user should be able to add books to the collection.
Adding a book means that the book metadata should be provided.
Books are added by manual entry -- the user fills out a form with a
predetermined set of fields, thereby populating the book metadata.

The application should be able to handle common user errors and missing data.

The program should be able to present the book collection to the user in a
table-like format, with one book per row. Each column displays one metadata
field and the columns are populated with metadata for the book at that row. 

The user should be able to delete books from the database by clicking a
suitably named button.


\subsubsection{Design}
Books are stored in a database using \texttt{JSON} serialization and basic
key/value lookup.

Each book has an unique id. No two books can have the same id.  The id can be
used as a handle or reference to a certain book stored in the database.

Each book is described by and stored with a set of metadata fields.  We are
provided with a set of required fields which will be used as a starting point:

\begin{itemize}
  \item Id
  \item Title
  \item Author
  \item genre
  \item Date of publication
  \item Price
  \item Description
\end{itemize}

The only field that must be present is the id.


\subsubsection{Extended functionality}
These are ideas on possible extra, ``wishlist'' functionality that could be
added in future releases.

I have previously written a program in Python that automatically renames e-books
from analysis and querying servers with \texttt{ISBN}-numbers to fill out
missing fields. Software for managing e-books often have this kind of
functionality, one example is the open source software package
``calibre''\cite{calibre}.

This program could also be extended to query services for any missing metadata
fields. I have some previous experience with automatic metadata extraction from
various documents and related metadata.

It also possible to add functionality for the program to populate all fields
automatically if the user simply enters a valid \texttt{ISBN} number.


% ______________________________________________________________________________
\subsection{Reflection on the creation of a Vision Document}\label{task-2-reflect}
I am used to documenting my own projects in this way. This includes both analog electronics,
embedded systems and software. The first thing I do when starting a project is to
sketch out what I imagine a working product could look like. I also like to pretend I'm using
the product, this often makes it easier to see which features would be useful. This goes
for both command-line programs and actual physical, analog signal processing units.

Sitting down and thinking hard about the product, its indented use and what it
would be like to interact with it, can be very revealing. A lot of less good
ideas can be weeded out quickly.

Also; at some level of complexity, there is simply no way to keep track of
everything without continuously documenting the development process.  After
some time away from a project, it can be very difficult to remember what was
going on.

The documentation also acts as a reminder as to which ideas sparked the initial
interest.  It is important to keep on track, especially when prone to
``over-engineering''\cite{wiki:overengineering} over let
``feature-creep''\cite{wiki:overengineering} hijack allocated time or budget.

Overall, I've found that documentation is key for a successful project.
Especially when working in a team, but also when working with your future self.

